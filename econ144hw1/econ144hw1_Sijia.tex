\documentclass[]{article}
\usepackage{lmodern}
\usepackage{amssymb,amsmath}
\usepackage{ifxetex,ifluatex}
\usepackage{fixltx2e} % provides \textsubscript
\ifnum 0\ifxetex 1\fi\ifluatex 1\fi=0 % if pdftex
  \usepackage[T1]{fontenc}
  \usepackage[utf8]{inputenc}
\else % if luatex or xelatex
  \ifxetex
    \usepackage{mathspec}
  \else
    \usepackage{fontspec}
  \fi
  \defaultfontfeatures{Ligatures=TeX,Scale=MatchLowercase}
\fi
% use upquote if available, for straight quotes in verbatim environments
\IfFileExists{upquote.sty}{\usepackage{upquote}}{}
% use microtype if available
\IfFileExists{microtype.sty}{%
\usepackage{microtype}
\UseMicrotypeSet[protrusion]{basicmath} % disable protrusion for tt fonts
}{}
\usepackage[margin=1in]{geometry}
\usepackage{hyperref}
\hypersetup{unicode=true,
            pdftitle={Econ144 hw1\_Sijia},
            pdfauthor={Sijia Hua},
            pdfborder={0 0 0},
            breaklinks=true}
\urlstyle{same}  % don't use monospace font for urls
\usepackage{color}
\usepackage{fancyvrb}
\newcommand{\VerbBar}{|}
\newcommand{\VERB}{\Verb[commandchars=\\\{\}]}
\DefineVerbatimEnvironment{Highlighting}{Verbatim}{commandchars=\\\{\}}
% Add ',fontsize=\small' for more characters per line
\usepackage{framed}
\definecolor{shadecolor}{RGB}{248,248,248}
\newenvironment{Shaded}{\begin{snugshade}}{\end{snugshade}}
\newcommand{\KeywordTok}[1]{\textcolor[rgb]{0.13,0.29,0.53}{\textbf{#1}}}
\newcommand{\DataTypeTok}[1]{\textcolor[rgb]{0.13,0.29,0.53}{#1}}
\newcommand{\DecValTok}[1]{\textcolor[rgb]{0.00,0.00,0.81}{#1}}
\newcommand{\BaseNTok}[1]{\textcolor[rgb]{0.00,0.00,0.81}{#1}}
\newcommand{\FloatTok}[1]{\textcolor[rgb]{0.00,0.00,0.81}{#1}}
\newcommand{\ConstantTok}[1]{\textcolor[rgb]{0.00,0.00,0.00}{#1}}
\newcommand{\CharTok}[1]{\textcolor[rgb]{0.31,0.60,0.02}{#1}}
\newcommand{\SpecialCharTok}[1]{\textcolor[rgb]{0.00,0.00,0.00}{#1}}
\newcommand{\StringTok}[1]{\textcolor[rgb]{0.31,0.60,0.02}{#1}}
\newcommand{\VerbatimStringTok}[1]{\textcolor[rgb]{0.31,0.60,0.02}{#1}}
\newcommand{\SpecialStringTok}[1]{\textcolor[rgb]{0.31,0.60,0.02}{#1}}
\newcommand{\ImportTok}[1]{#1}
\newcommand{\CommentTok}[1]{\textcolor[rgb]{0.56,0.35,0.01}{\textit{#1}}}
\newcommand{\DocumentationTok}[1]{\textcolor[rgb]{0.56,0.35,0.01}{\textbf{\textit{#1}}}}
\newcommand{\AnnotationTok}[1]{\textcolor[rgb]{0.56,0.35,0.01}{\textbf{\textit{#1}}}}
\newcommand{\CommentVarTok}[1]{\textcolor[rgb]{0.56,0.35,0.01}{\textbf{\textit{#1}}}}
\newcommand{\OtherTok}[1]{\textcolor[rgb]{0.56,0.35,0.01}{#1}}
\newcommand{\FunctionTok}[1]{\textcolor[rgb]{0.00,0.00,0.00}{#1}}
\newcommand{\VariableTok}[1]{\textcolor[rgb]{0.00,0.00,0.00}{#1}}
\newcommand{\ControlFlowTok}[1]{\textcolor[rgb]{0.13,0.29,0.53}{\textbf{#1}}}
\newcommand{\OperatorTok}[1]{\textcolor[rgb]{0.81,0.36,0.00}{\textbf{#1}}}
\newcommand{\BuiltInTok}[1]{#1}
\newcommand{\ExtensionTok}[1]{#1}
\newcommand{\PreprocessorTok}[1]{\textcolor[rgb]{0.56,0.35,0.01}{\textit{#1}}}
\newcommand{\AttributeTok}[1]{\textcolor[rgb]{0.77,0.63,0.00}{#1}}
\newcommand{\RegionMarkerTok}[1]{#1}
\newcommand{\InformationTok}[1]{\textcolor[rgb]{0.56,0.35,0.01}{\textbf{\textit{#1}}}}
\newcommand{\WarningTok}[1]{\textcolor[rgb]{0.56,0.35,0.01}{\textbf{\textit{#1}}}}
\newcommand{\AlertTok}[1]{\textcolor[rgb]{0.94,0.16,0.16}{#1}}
\newcommand{\ErrorTok}[1]{\textcolor[rgb]{0.64,0.00,0.00}{\textbf{#1}}}
\newcommand{\NormalTok}[1]{#1}
\usepackage{graphicx,grffile}
\makeatletter
\def\maxwidth{\ifdim\Gin@nat@width>\linewidth\linewidth\else\Gin@nat@width\fi}
\def\maxheight{\ifdim\Gin@nat@height>\textheight\textheight\else\Gin@nat@height\fi}
\makeatother
% Scale images if necessary, so that they will not overflow the page
% margins by default, and it is still possible to overwrite the defaults
% using explicit options in \includegraphics[width, height, ...]{}
\setkeys{Gin}{width=\maxwidth,height=\maxheight,keepaspectratio}
\IfFileExists{parskip.sty}{%
\usepackage{parskip}
}{% else
\setlength{\parindent}{0pt}
\setlength{\parskip}{6pt plus 2pt minus 1pt}
}
\setlength{\emergencystretch}{3em}  % prevent overfull lines
\providecommand{\tightlist}{%
  \setlength{\itemsep}{0pt}\setlength{\parskip}{0pt}}
\setcounter{secnumdepth}{0}
% Redefines (sub)paragraphs to behave more like sections
\ifx\paragraph\undefined\else
\let\oldparagraph\paragraph
\renewcommand{\paragraph}[1]{\oldparagraph{#1}\mbox{}}
\fi
\ifx\subparagraph\undefined\else
\let\oldsubparagraph\subparagraph
\renewcommand{\subparagraph}[1]{\oldsubparagraph{#1}\mbox{}}
\fi

%%% Use protect on footnotes to avoid problems with footnotes in titles
\let\rmarkdownfootnote\footnote%
\def\footnote{\protect\rmarkdownfootnote}

%%% Change title format to be more compact
\usepackage{titling}

% Create subtitle command for use in maketitle
\providecommand{\subtitle}[1]{
  \posttitle{
    \begin{center}\large#1\end{center}
    }
}

\setlength{\droptitle}{-2em}

  \title{Econ144 hw1\_Sijia}
    \pretitle{\vspace{\droptitle}\centering\huge}
  \posttitle{\par}
    \author{Sijia Hua}
    \preauthor{\centering\large\emph}
  \postauthor{\par}
      \predate{\centering\large\emph}
  \postdate{\par}
    \date{4/8/2019}


\begin{document}
\maketitle

\subsection{Econ 144 HW1}\label{econ-144-hw1}

\subsubsection{Chapter 2 \#2}\label{chapter-2-2}

\begin{Shaded}
\begin{Highlighting}[]
\KeywordTok{setwd}\NormalTok{(}\StringTok{"/Users/Renaissance/Desktop"}\NormalTok{)}
\NormalTok{hw11<-}\KeywordTok{read.xls}\NormalTok{(}\StringTok{'econ144hw1data.xlsx'}\NormalTok{,}\DataTypeTok{sheet =} \DecValTok{1}\NormalTok{) }\CommentTok{#import data}
\end{Highlighting}
\end{Shaded}

\begin{Shaded}
\begin{Highlighting}[]
\CommentTok{# compute descriptive statistics }
\NormalTok{ hw11 }\OperatorTok\StringTok{ }\KeywordTok{select}\NormalTok{(}\KeywordTok{c}\NormalTok{(}\OperatorTok{-}\DecValTok{1}\NormalTok{)) }\OperatorTok\StringTok{ }\KeywordTok{summary}\NormalTok{()}
\end{Highlighting}
\end{Shaded}

\begin{verbatim}
##      GRGDP             RETURN       
##  Min.   :-2.7075   Min.   :-26.937  
##  1st Qu.: 0.3273   1st Qu.: -0.915  
##  Median : 0.7679   Median :  2.023  
##  Mean   : 0.7958   Mean   :  1.962  
##  3rd Qu.: 1.3107   3rd Qu.:  5.753  
##  Max.   : 3.9343   Max.   : 20.117
\end{verbatim}

\begin{Shaded}
\begin{Highlighting}[]
\KeywordTok{paste}\NormalTok{(}\StringTok{"The skewness of GRGDP is "}\NormalTok{, }\KeywordTok{round}\NormalTok{(}\KeywordTok{skewness}\NormalTok{(hw11}\OperatorTok{$}\NormalTok{GRGDP),}\DecValTok{3}\NormalTok{),}\StringTok{". Hence GRGDP is left-skewness."}\NormalTok{) }
\end{Highlighting}
\end{Shaded}

\begin{verbatim}
## [1] "The skewness of GRGDP is  -0.189 . Hence GRGDP is left-skewness."
\end{verbatim}

\begin{Shaded}
\begin{Highlighting}[]
\KeywordTok{paste}\NormalTok{(}\StringTok{"The skewness of Return is "}\NormalTok{, }\KeywordTok{round}\NormalTok{(}\KeywordTok{skewness}\NormalTok{(hw11}\OperatorTok{$}\NormalTok{RETURN),}\DecValTok{3}\NormalTok{),}\StringTok{". Hence Return is left-skewness."}\NormalTok{) }
\end{Highlighting}
\end{Shaded}

\begin{verbatim}
## [1] "The skewness of Return is  -0.684 . Hence Return is left-skewness."
\end{verbatim}

\begin{Shaded}
\begin{Highlighting}[]
\KeywordTok{paste}\NormalTok{(}\StringTok{"The kurtosis of GRGDP is "}\NormalTok{,}\KeywordTok{round}\NormalTok{(}\KeywordTok{kurtosis}\NormalTok{(hw11}\OperatorTok{$}\NormalTok{GRGDP),}\DecValTok{3}\NormalTok{))}
\end{Highlighting}
\end{Shaded}

\begin{verbatim}
## [1] "The kurtosis of GRGDP is  4.418"
\end{verbatim}

\begin{Shaded}
\begin{Highlighting}[]
\KeywordTok{paste}\NormalTok{(}\StringTok{"The kurtosis of Return is "}\NormalTok{,}\KeywordTok{round}\NormalTok{(}\KeywordTok{kurtosis}\NormalTok{(hw11}\OperatorTok{$}\NormalTok{RETURN),}\DecValTok{3}\NormalTok{))}
\end{Highlighting}
\end{Shaded}

\begin{verbatim}
## [1] "The kurtosis of Return is  5.271"
\end{verbatim}

\begin{Shaded}
\begin{Highlighting}[]
\KeywordTok{paste}\NormalTok{(}\StringTok{"The Jarque-Bera Test of GRGDP is "}\NormalTok{)}
\end{Highlighting}
\end{Shaded}

\begin{verbatim}
## [1] "The Jarque-Bera Test of GRGDP is "
\end{verbatim}

\begin{Shaded}
\begin{Highlighting}[]
\KeywordTok{jarque.test}\NormalTok{(hw11}\OperatorTok{$}\NormalTok{GRGDP)}
\end{Highlighting}
\end{Shaded}

\begin{verbatim}
## 
##  Jarque-Bera Normality Test
## 
## data:  hw11$GRGDP
## JB = 22.265, p-value = 1.463e-05
## alternative hypothesis: greater
\end{verbatim}

\begin{Shaded}
\begin{Highlighting}[]
\KeywordTok{paste}\NormalTok{(}\StringTok{"The Jarque-Bera Test of Return is "}\NormalTok{)}
\end{Highlighting}
\end{Shaded}

\begin{verbatim}
## [1] "The Jarque-Bera Test of Return is "
\end{verbatim}

\begin{Shaded}
\begin{Highlighting}[]
\KeywordTok{jarque.test}\NormalTok{(hw11}\OperatorTok{$}\NormalTok{RETURN)}
\end{Highlighting}
\end{Shaded}

\begin{verbatim}
## 
##  Jarque-Bera Normality Test
## 
## data:  hw11$RETURN
## JB = 72.627, p-value = 2.22e-16
## alternative hypothesis: greater
\end{verbatim}

\begin{Shaded}
\begin{Highlighting}[]
\KeywordTok{hist}\NormalTok{(hw11}\OperatorTok{$}\NormalTok{GRGDP,}\DataTypeTok{main=}\StringTok{"Histogram of GRGDP"}\NormalTok{,}\DataTypeTok{xlab=}\StringTok{"GRGDP"}\NormalTok{)}
\end{Highlighting}
\end{Shaded}

\includegraphics{econ144hw1_Sijia_files/figure-latex/unnamed-chunk-1-1.pdf}

\begin{Shaded}
\begin{Highlighting}[]
\KeywordTok{hist}\NormalTok{(hw11}\OperatorTok{$}\NormalTok{RETURN, }\DataTypeTok{main=}\StringTok{"Histogram of SP500 Quarterly Return"}\NormalTok{,}\DataTypeTok{xlab=}\StringTok{"Return"}\NormalTok{)}
\end{Highlighting}
\end{Shaded}

\includegraphics{econ144hw1_Sijia_files/figure-latex/unnamed-chunk-1-2.pdf}

\begin{Shaded}
\begin{Highlighting}[]
\KeywordTok{boxplot}\NormalTok{(hw11}\OperatorTok{$}\NormalTok{GRGDP,}\DataTypeTok{main=}\StringTok{"Box Plot of GRGDP"}\NormalTok{,}\DataTypeTok{horizontal =} \OtherTok{TRUE}\NormalTok{)}
\end{Highlighting}
\end{Shaded}

\includegraphics{econ144hw1_Sijia_files/figure-latex/boxplot in 2-1.pdf}

\begin{Shaded}
\begin{Highlighting}[]
\KeywordTok{boxplot}\NormalTok{(hw11}\OperatorTok{$}\NormalTok{RETURN,}\DataTypeTok{main=}\StringTok{"Box Plot of Return"}\NormalTok{,}\DataTypeTok{horizontal =} \OtherTok{TRUE}\NormalTok{)}
\end{Highlighting}
\end{Shaded}

\includegraphics{econ144hw1_Sijia_files/figure-latex/boxplot in 2-2.pdf}

\begin{Shaded}
\begin{Highlighting}[]
\CommentTok{# ls <- lm(hw12$GRGDP~hw12$RETURN) #find linear regression between GDP and returns}
\CommentTok{# summary(ls)}
\KeywordTok{paste}\NormalTok{(}\StringTok{"Correlation betwen this two variable is "}\NormalTok{,}\KeywordTok{round}\NormalTok{(}\KeywordTok{cor}\NormalTok{(hw11}\OperatorTok{$}\NormalTok{RETURN,hw11}\OperatorTok{$}\NormalTok{GRGDP,}\DataTypeTok{method=}\StringTok{"pearson"}\NormalTok{),}\DecValTok{3}\NormalTok{), }\StringTok{". This indicates that GRGDP has a low correlation with Return."}\NormalTok{)}
\end{Highlighting}
\end{Shaded}

\begin{verbatim}
## [1] "Correlation betwen this two variable is  0.27 . This indicates that GRGDP has a low correlation with Return."
\end{verbatim}

\subsubsection{Chapter 2 \#4}\label{chapter-2-4}

\begin{Shaded}
\begin{Highlighting}[]
\CommentTok{# still use hw11 data}
\NormalTok{y=}\KeywordTok{ts}\NormalTok{(hw11}\OperatorTok{$}\NormalTok{GRGDP,}\DataTypeTok{start=}\FloatTok{1950.25}\NormalTok{, }\DataTypeTok{freq=}\DecValTok{1}\NormalTok{)}
\NormalTok{x=}\KeywordTok{ts}\NormalTok{(hw11}\OperatorTok{$}\NormalTok{RETURN,}\DataTypeTok{start=}\FloatTok{1950.25}\NormalTok{, }\DataTypeTok{freq=}\DecValTok{1}\NormalTok{)}

\CommentTok{# (a) }
\NormalTok{ma=}\KeywordTok{lm}\NormalTok{(y}\OperatorTok{~}\NormalTok{x)}
\KeywordTok{summary}\NormalTok{(ma)}
\end{Highlighting}
\end{Shaded}

\begin{verbatim}
## 
## Call:
## lm(formula = y ~ x)
## 
## Residuals:
##     Min      1Q  Median      3Q     Max 
## -3.5068 -0.5271 -0.0500  0.5344  3.2189 
## 
## Coefficients:
##             Estimate Std. Error t value Pr(>|t|)    
## (Intercept) 0.710672   0.062819  11.313  < 2e-16 ***
## x           0.043388   0.009856   4.402  1.6e-05 ***
## ---
## Signif. codes:  0 '***' 0.001 '**' 0.01 '*' 0.05 '.' 0.1 ' ' 1
## 
## Residual standard error: 0.9412 on 246 degrees of freedom
## Multiple R-squared:  0.07303,    Adjusted R-squared:  0.06926 
## F-statistic: 19.38 on 1 and 246 DF,  p-value: 1.597e-05
\end{verbatim}

\indent I choose the significant level = 0.05. As the p-value is
1.6e-05, much less than 0.05, we reject the null hypothesis that β = 0.
Hence, there is a (significant) relationship between the variables in
the linear regression model of the data set. In this case, stock market
is a leading indicator.

\begin{Shaded}
\begin{Highlighting}[]
\CommentTok{# (b) }
\NormalTok{mb=}\KeywordTok{dynlm}\NormalTok{(y}\OperatorTok{~}\KeywordTok{L}\NormalTok{(x,}\DecValTok{1}\NormalTok{))}
\KeywordTok{summary}\NormalTok{(mb)    }\CommentTok{#r-square has improved }
\end{Highlighting}
\end{Shaded}

\begin{verbatim}
## 
## Time series regression with "ts" data:
## Start = 1951, End = 2197
## 
## Call:
## dynlm(formula = y ~ L(x, 1))
## 
## Residuals:
##     Min      1Q  Median      3Q     Max 
## -3.0231 -0.5339 -0.0346  0.4720  3.5844 
## 
## Coefficients:
##             Estimate Std. Error t value Pr(>|t|)    
## (Intercept) 0.661818   0.059134  11.192  < 2e-16 ***
## L(x, 1)     0.064728   0.009305   6.956 3.17e-11 ***
## ---
## Signif. codes:  0 '***' 0.001 '**' 0.01 '*' 0.05 '.' 0.1 ' ' 1
## 
## Residual standard error: 0.8855 on 245 degrees of freedom
## Multiple R-squared:  0.1649, Adjusted R-squared:  0.1615 
## F-statistic: 48.39 on 1 and 245 DF,  p-value: 3.169e-11
\end{verbatim}

\indent I choose the significant level = 0.05. As the p-value is
3.17e-11, much less than 0.05, we reject the null hypothesis that β = 0.
Hence, there is a (significant) relationship between the variables in
the linear regression model of the data set. In this case, stock market
is a leading indicator.

\subsubsection{Chapter 2 \#5}\label{chapter-2-5}

\begin{Shaded}
\begin{Highlighting}[]
\CommentTok{# (c) }
\NormalTok{mc=}\KeywordTok{dynlm}\NormalTok{(y}\OperatorTok{~}\KeywordTok{L}\NormalTok{(x,}\DecValTok{1}\NormalTok{)}\OperatorTok{+}\KeywordTok{L}\NormalTok{(x,}\DecValTok{2}\NormalTok{)}\OperatorTok{+}\KeywordTok{L}\NormalTok{(x,}\DecValTok{3}\NormalTok{)}\OperatorTok{+}\KeywordTok{L}\NormalTok{(x,}\DecValTok{4}\NormalTok{))}
\KeywordTok{summary}\NormalTok{(mc)}
\end{Highlighting}
\end{Shaded}

\begin{verbatim}
## 
## Time series regression with "ts" data:
## Start = 1954, End = 2197
## 
## Call:
## dynlm(formula = y ~ L(x, 1) + L(x, 2) + L(x, 3) + L(x, 4))
## 
## Residuals:
##     Min      1Q  Median      3Q     Max 
## -2.9786 -0.5113  0.0068  0.4842  3.7592 
## 
## Coefficients:
##             Estimate Std. Error t value Pr(>|t|)    
## (Intercept) 0.571948   0.061504   9.299  < 2e-16 ***
## L(x, 1)     0.056594   0.009735   5.814 1.95e-08 ***
## L(x, 2)     0.018011   0.010471   1.720   0.0867 .  
## L(x, 3)     0.015672   0.010493   1.494   0.1366    
## L(x, 4)     0.011948   0.009719   1.229   0.2201    
## ---
## Signif. codes:  0 '***' 0.001 '**' 0.01 '*' 0.05 '.' 0.1 ' ' 1
## 
## Residual standard error: 0.8526 on 239 degrees of freedom
## Multiple R-squared:  0.2066, Adjusted R-squared:  0.1933 
## F-statistic: 15.56 on 4 and 239 DF,  p-value: 2.509e-11
\end{verbatim}

\indent  In summary, The p value of L(x,1) is 1.95e-08, which is smaller
than 0.05. We reject H0: \beta\_1 =0. The stock market with 1-quarter
leading is an indicator. The p value of L(x,2) is 0.0867, which is
bigger than 0.05. We can not reject H0. There is no significant
difference. Stock market's 2-quarter leading price may not be an
indicator. The p value of L(x,3) and L(x,4) is 0.1366 and 0.2201
respectively, which are bigger than 0.05. We can not reject H0.There is
no significant difference. \indent P value of the F-statistic is
2.509e-11, which is smaller than 0.05.Hence, we can reject H0, the
overall addition of the variables is significantly improving the model.

\begin{Shaded}
\begin{Highlighting}[]
\NormalTok{md=}\KeywordTok{dynlm}\NormalTok{(y}\OperatorTok{~}\KeywordTok{L}\NormalTok{(x,}\DecValTok{1}\NormalTok{)}\OperatorTok{+}\KeywordTok{L}\NormalTok{(x,}\DecValTok{2}\NormalTok{)}\OperatorTok{+}\KeywordTok{L}\NormalTok{(x,}\DecValTok{3}\NormalTok{)}\OperatorTok{+}\KeywordTok{L}\NormalTok{(x,}\DecValTok{4}\NormalTok{)}\OperatorTok{+}\KeywordTok{L}\NormalTok{(y,}\DecValTok{1}\NormalTok{))}
\KeywordTok{summary}\NormalTok{(md)}
\end{Highlighting}
\end{Shaded}

\begin{verbatim}
## 
## Time series regression with "ts" data:
## Start = 1954, End = 2197
## 
## Call:
## dynlm(formula = y ~ L(x, 1) + L(x, 2) + L(x, 3) + L(x, 4) + L(y, 
##     1))
## 
## Residuals:
##     Min      1Q  Median      3Q     Max 
## -2.8733 -0.4469 -0.0007  0.5037  3.7168 
## 
## Coefficients:
##             Estimate Std. Error t value Pr(>|t|)    
## (Intercept) 0.444490   0.069652   6.382 9.07e-10 ***
## L(x, 1)     0.050535   0.009646   5.239 3.56e-07 ***
## L(x, 2)     0.007459   0.010629   0.702 0.483503    
## L(x, 3)     0.011149   0.010316   1.081 0.280918    
## L(x, 4)     0.007133   0.009577   0.745 0.457103    
## L(y, 1)     0.230396   0.063898   3.606 0.000379 ***
## ---
## Signif. codes:  0 '***' 0.001 '**' 0.01 '*' 0.05 '.' 0.1 ' ' 1
## 
## Residual standard error: 0.832 on 238 degrees of freedom
## Multiple R-squared:  0.2477, Adjusted R-squared:  0.2319 
## F-statistic: 15.67 on 5 and 238 DF,  p-value: 2.483e-13
\end{verbatim}

\subsubsection{Chapter 2 \#7}\label{chapter-2-7}

\begin{Shaded}
\begin{Highlighting}[]
\NormalTok{hw12<-}\KeywordTok{read.xls}\NormalTok{(}\StringTok{'econ144hw1data.xlsx'}\NormalTok{,}\DataTypeTok{sheet =} \DecValTok{2}\NormalTok{) }\CommentTok{# read data}

\CommentTok{# calculate growth rate}
\NormalTok{grate_unem <-}\StringTok{ }\ControlFlowTok{function}\NormalTok{(x)(x}\OperatorTok{/}\KeywordTok{lag}\NormalTok{(x)}\OperatorTok{-}\DecValTok{1}\NormalTok{)}\OperatorTok{*}\DecValTok{100}
\NormalTok{hw12}\OperatorTok{$}\NormalTok{growth_unem <-}\KeywordTok{round}\NormalTok{(}\KeywordTok{grate_unem}\NormalTok{(hw12}\OperatorTok{$}\NormalTok{UNEM),}\DecValTok{3}\NormalTok{)}
\NormalTok{grate_pov <-}\StringTok{ }\ControlFlowTok{function}\NormalTok{(x)(x}\OperatorTok{/}\KeywordTok{lag}\NormalTok{(x)}\OperatorTok{-}\DecValTok{1}\NormalTok{)}\OperatorTok{*}\DecValTok{100}
\NormalTok{hw12}\OperatorTok{$}\NormalTok{growth_pov <-}\StringTok{ }\KeywordTok{round}\NormalTok{(}\KeywordTok{grate_pov}\NormalTok{(hw12}\OperatorTok{$}\NormalTok{POV),}\DecValTok{3}\NormalTok{)}

\CommentTok{# delete blank column}
\NormalTok{hw12<-hw12}\OperatorTok\StringTok{ }\KeywordTok{select}\NormalTok{(}\KeywordTok{c}\NormalTok{(}\OperatorTok{-}\DecValTok{4}\NormalTok{))}

\CommentTok{# drop first row (NA)}
\NormalTok{newhw12 <-}\StringTok{ }\NormalTok{hw12 }\OperatorTok\StringTok{ }\KeywordTok{drop_na}\NormalTok{()}
\end{Highlighting}
\end{Shaded}

\subsubsection{descriptive statistics of two growth
rates}\label{descriptive-statistics-of-two-growth-rates}

\begin{Shaded}
\begin{Highlighting}[]
\CommentTok{# descriptive statistic}
\KeywordTok{summary}\NormalTok{(hw12)}
\end{Highlighting}
\end{Shaded}

\begin{verbatim}
##       year           POV             UNEM        growth_unem     
##  Min.   :1959   Min.   :22973   Min.   : 2797   Min.   :-20.245  
##  1st Qu.:1972   1st Qu.:28325   1st Qu.: 4834   1st Qu.: -7.125  
##  Median :1984   Median :33642   Median : 6979   Median : -2.376  
##  Mean   :1984   Mean   :32982   Mean   : 6795   Mean   :  4.004  
##  3rd Qu.:1997   3rd Qu.:36634   3rd Qu.: 8253   3rd Qu.:  9.735  
##  Max.   :2010   Max.   :46180   Max.   :14815   Max.   : 59.770  
##                                                 NA's   :1        
##    growth_pov      
##  Min.   :-14.0880  
##  1st Qu.: -3.0790  
##  Median : -0.5600  
##  Mean   :  0.4489  
##  3rd Qu.:  4.6265  
##  Max.   : 12.2740  
##  NA's   :1
\end{verbatim}

\begin{Shaded}
\begin{Highlighting}[]
\KeywordTok{paste}\NormalTok{(}\StringTok{"The skewness of growth rate of unemployed persons is "}\NormalTok{, }\KeywordTok{round}\NormalTok{(}\KeywordTok{skewness}\NormalTok{(newhw12}\OperatorTok{$}\NormalTok{growth_unem),}\DecValTok{3}\NormalTok{),}\StringTok{". Hence the growth rate of unemployed person is right-skewness."}\NormalTok{) }
\end{Highlighting}
\end{Shaded}

\begin{verbatim}
## [1] "The skewness of growth rate of unemployed persons is  1.425 . Hence the growth rate of unemployed person is right-skewness."
\end{verbatim}

\begin{Shaded}
\begin{Highlighting}[]
\KeywordTok{paste}\NormalTok{(}\StringTok{"The skewness of growth rate of number of people in poverty is "}\NormalTok{, }\KeywordTok{round}\NormalTok{(}\KeywordTok{skewness}\NormalTok{(newhw12}\OperatorTok{$}\NormalTok{growth_pov),}\DecValTok{3}\NormalTok{),}\StringTok{". Hence Return is right-skewness."}\NormalTok{) }
\end{Highlighting}
\end{Shaded}

\begin{verbatim}
## [1] "The skewness of growth rate of number of people in poverty is  0.027 . Hence Return is right-skewness."
\end{verbatim}

\begin{Shaded}
\begin{Highlighting}[]
\KeywordTok{paste}\NormalTok{(}\StringTok{"The kurtosis of growth rate of unemployed person is "}\NormalTok{,}\KeywordTok{round}\NormalTok{(}\KeywordTok{kurtosis}\NormalTok{(newhw12}\OperatorTok{$}\NormalTok{growth_unem),}\DecValTok{3}\NormalTok{))}
\end{Highlighting}
\end{Shaded}

\begin{verbatim}
## [1] "The kurtosis of growth rate of unemployed person is  4.74"
\end{verbatim}

\begin{Shaded}
\begin{Highlighting}[]
\KeywordTok{paste}\NormalTok{(}\StringTok{"The kurtosis of growth rate of number of people in poverty is "}\NormalTok{,}\KeywordTok{round}\NormalTok{(}\KeywordTok{kurtosis}\NormalTok{(newhw12}\OperatorTok{$}\NormalTok{growth_pov),}\DecValTok{3}\NormalTok{))}
\end{Highlighting}
\end{Shaded}

\begin{verbatim}
## [1] "The kurtosis of growth rate of number of people in poverty is  2.885"
\end{verbatim}

\begin{Shaded}
\begin{Highlighting}[]
\KeywordTok{paste}\NormalTok{(}\StringTok{"The Jarque-Bera Test of growth rate of unemployed person is "}\NormalTok{)}
\end{Highlighting}
\end{Shaded}

\begin{verbatim}
## [1] "The Jarque-Bera Test of growth rate of unemployed person is "
\end{verbatim}

\begin{Shaded}
\begin{Highlighting}[]
\KeywordTok{jarque.test}\NormalTok{(newhw12}\OperatorTok{$}\NormalTok{growth_unem)}
\end{Highlighting}
\end{Shaded}

\begin{verbatim}
## 
##  Jarque-Bera Normality Test
## 
## data:  newhw12$growth_unem
## JB = 23.703, p-value = 7.127e-06
## alternative hypothesis: greater
\end{verbatim}

\begin{Shaded}
\begin{Highlighting}[]
\KeywordTok{paste}\NormalTok{(}\StringTok{"The Jarque-Bera Test of growth rate of number of people is "}\NormalTok{)}
\end{Highlighting}
\end{Shaded}

\begin{verbatim}
## [1] "The Jarque-Bera Test of growth rate of number of people is "
\end{verbatim}

\begin{Shaded}
\begin{Highlighting}[]
\KeywordTok{jarque.test}\NormalTok{(newhw12}\OperatorTok{$}\NormalTok{growth_pov)}
\end{Highlighting}
\end{Shaded}

\begin{verbatim}
## 
##  Jarque-Bera Normality Test
## 
## data:  newhw12$growth_pov
## JB = 0.034273, p-value = 0.983
## alternative hypothesis: greater
\end{verbatim}

\begin{Shaded}
\begin{Highlighting}[]
\KeywordTok{hist}\NormalTok{(hw12}\OperatorTok{$}\NormalTok{growth_unem,}\DataTypeTok{main=}\StringTok{"Histogram of growth rate of unemployed perons"}\NormalTok{,}\DataTypeTok{xlab=}\StringTok{"Growth rate of unemployment"}\NormalTok{)}
\end{Highlighting}
\end{Shaded}

\includegraphics{econ144hw1_Sijia_files/figure-latex/histogram-1.pdf}

\begin{Shaded}
\begin{Highlighting}[]
\KeywordTok{hist}\NormalTok{(hw12}\OperatorTok{$}\NormalTok{growth_pov, }\DataTypeTok{main=}\StringTok{"Histogram of growth rate of number of people in poverty"}\NormalTok{,}\DataTypeTok{xlab=}\StringTok{"Growth rate of people in poverty"}\NormalTok{)}
\end{Highlighting}
\end{Shaded}

\includegraphics{econ144hw1_Sijia_files/figure-latex/histogram-2.pdf}

\begin{Shaded}
\begin{Highlighting}[]
\KeywordTok{boxplot}\NormalTok{(hw12}\OperatorTok{$}\NormalTok{growth_unem,}\DataTypeTok{main=}\StringTok{"Box Plot of growth rate of unemployment"}\NormalTok{,}\DataTypeTok{horizontal =} \OtherTok{TRUE}\NormalTok{)}
\end{Highlighting}
\end{Shaded}

\includegraphics{econ144hw1_Sijia_files/figure-latex/boxplot-1.pdf}

\begin{Shaded}
\begin{Highlighting}[]
\KeywordTok{boxplot}\NormalTok{(hw12}\OperatorTok{$}\NormalTok{growth_pov,}\DataTypeTok{main=}\StringTok{"Box Plot of growth rate of poverty"}\NormalTok{,}\DataTypeTok{horizontal =} \OtherTok{TRUE}\NormalTok{)}
\end{Highlighting}
\end{Shaded}

\includegraphics{econ144hw1_Sijia_files/figure-latex/boxplot-2.pdf}

\begin{Shaded}
\begin{Highlighting}[]
\KeywordTok{plot}\NormalTok{(hw12}\OperatorTok{$}\NormalTok{year,hw12}\OperatorTok{$}\NormalTok{growth_unem,}\DataTypeTok{type=}\StringTok{"l"}\NormalTok{,}\DataTypeTok{lwd =} \FloatTok{1.5}\NormalTok{,}\DataTypeTok{xlab =} \StringTok{"year"}\NormalTok{,}\DataTypeTok{ylab=}\StringTok{"Growth rate"}\NormalTok{)}
\KeywordTok{lines}\NormalTok{(hw12}\OperatorTok{$}\NormalTok{year,hw12}\OperatorTok{$}\NormalTok{growth_pov,}\DataTypeTok{type=}\StringTok{"l"}\NormalTok{,}\DataTypeTok{lwd=}\FloatTok{1.5}\NormalTok{, }\DataTypeTok{col=}\StringTok{"blue"}\NormalTok{)}
\KeywordTok{legend}\NormalTok{(}\StringTok{"topleft"}\NormalTok{, }\DataTypeTok{legend =} \KeywordTok{c}\NormalTok{(}\StringTok{"g.r.of unemployment"}\NormalTok{, }\StringTok{"g.r.of poverty"}\NormalTok{), }\DataTypeTok{col =} \KeywordTok{c}\NormalTok{(}\StringTok{"black"}\NormalTok{,}\StringTok{"blue"}\NormalTok{), }\DataTypeTok{pch =} \KeywordTok{c}\NormalTok{(}\DecValTok{1}\NormalTok{,}\DecValTok{1}\NormalTok{), }\DataTypeTok{bty =} \StringTok{"n"}\NormalTok{, }\DataTypeTok{pt.cex =} \DecValTok{1}\NormalTok{, }\DataTypeTok{cex =} \DecValTok{1}\NormalTok{, }\DataTypeTok{text.col =} \StringTok{"black"}\NormalTok{, }\DataTypeTok{horiz =}\NormalTok{ F)}
\end{Highlighting}
\end{Shaded}

\includegraphics{econ144hw1_Sijia_files/figure-latex/correlation-1.pdf}

\begin{Shaded}
\begin{Highlighting}[]
\KeywordTok{paste}\NormalTok{(}\StringTok{"Correlation betwen this two variable is "}\NormalTok{,}\KeywordTok{round}\NormalTok{(}\KeywordTok{cor}\NormalTok{(hw11}\OperatorTok{$}\NormalTok{RETURN,hw11}\OperatorTok{$}\NormalTok{GRGDP,}\DataTypeTok{method=}\StringTok{"pearson"}\NormalTok{),}\DecValTok{3}\NormalTok{), }\StringTok{". This indicates that GRGDP has a low correlation with Return."}\NormalTok{)}
\end{Highlighting}
\end{Shaded}

\begin{verbatim}
## [1] "Correlation betwen this two variable is  0.27 . This indicates that GRGDP has a low correlation with Return."
\end{verbatim}

\subsubsection{Chapter 2 \#8}\label{chapter-2-8}

\begin{Shaded}
\begin{Highlighting}[]
\NormalTok{hw13<-}\KeywordTok{read.xls}\NormalTok{(}\StringTok{'econ144hw1data.xlsx'}\NormalTok{,}\DataTypeTok{sheet =} \DecValTok{3}\NormalTok{) }\CommentTok{# read data}
\NormalTok{y=}\KeywordTok{ts}\NormalTok{(hw13}\OperatorTok{$}\NormalTok{G_POV,}\DataTypeTok{start=}\DecValTok{1960}\NormalTok{, }\DataTypeTok{freq=}\DecValTok{1}\NormalTok{)}
\NormalTok{x=}\KeywordTok{ts}\NormalTok{(hw13}\OperatorTok{$}\NormalTok{G_UNEM,}\DataTypeTok{start=}\DecValTok{1960}\NormalTok{, }\DataTypeTok{freq=}\DecValTok{1}\NormalTok{)}
\end{Highlighting}
\end{Shaded}

\begin{Shaded}
\begin{Highlighting}[]
\CommentTok{# (a) }
\NormalTok{modela=}\KeywordTok{lm}\NormalTok{(y}\OperatorTok{~}\NormalTok{x)}
\KeywordTok{summary}\NormalTok{(modela)}
\end{Highlighting}
\end{Shaded}

\begin{verbatim}
## 
## Call:
## lm(formula = y ~ x)
## 
## Residuals:
##      Min       1Q   Median       3Q      Max 
## -10.4671  -2.7537   0.3376   2.0197   7.4094 
## 
## Coefficients:
##             Estimate Std. Error t value Pr(>|t|)    
## (Intercept)  -0.4307     0.5498  -0.783    0.437    
## x             0.2197     0.0311   7.063  5.3e-09 ***
## ---
## Signif. codes:  0 '***' 0.001 '**' 0.01 '*' 0.05 '.' 0.1 ' ' 1
## 
## Residual standard error: 3.825 on 49 degrees of freedom
## Multiple R-squared:  0.5045, Adjusted R-squared:  0.4944 
## F-statistic: 49.89 on 1 and 49 DF,  p-value: 5.296e-09
\end{verbatim}

For model a, multiple R-squared is 0.5045, adjusted R-squared is 0.4944.

\begin{Shaded}
\begin{Highlighting}[]
\CommentTok{# I choose model b,c,d in excercise #3}

\CommentTok{# (b) }
\NormalTok{modelb=}\KeywordTok{dynlm}\NormalTok{(y}\OperatorTok{~}\KeywordTok{L}\NormalTok{(x,}\DecValTok{1}\NormalTok{))}
\KeywordTok{summary}\NormalTok{(modelb)}
\end{Highlighting}
\end{Shaded}

\begin{verbatim}
## 
## Time series regression with "ts" data:
## Start = 1961, End = 2010
## 
## Call:
## dynlm(formula = y ~ L(x, 1))
## 
## Residuals:
##      Min       1Q   Median       3Q      Max 
## -12.8549  -3.1164  -0.3667   3.0453  12.3684 
## 
## Coefficients:
##              Estimate Std. Error t value Pr(>|t|)  
## (Intercept) -0.002186   0.744465  -0.003   0.9977  
## L(x, 1)      0.110109   0.041716   2.640   0.0112 *
## ---
## Signif. codes:  0 '***' 0.001 '**' 0.01 '*' 0.05 '.' 0.1 ' ' 1
## 
## Residual standard error: 5.129 on 48 degrees of freedom
## Multiple R-squared:  0.1267, Adjusted R-squared:  0.1086 
## F-statistic: 6.967 on 1 and 48 DF,  p-value: 0.01116
\end{verbatim}

\indent For model b, multiple R-squared is 0.1267, adjusted r-squared is
0.1086.

\begin{Shaded}
\begin{Highlighting}[]
\CommentTok{# (c) }
\NormalTok{modelc=}\KeywordTok{dynlm}\NormalTok{(y}\OperatorTok{~}\KeywordTok{L}\NormalTok{(x,}\DecValTok{1}\NormalTok{)}\OperatorTok{+}\KeywordTok{L}\NormalTok{(x,}\DecValTok{2}\NormalTok{)}\OperatorTok{+}\KeywordTok{L}\NormalTok{(x,}\DecValTok{3}\NormalTok{)}\OperatorTok{+}\KeywordTok{L}\NormalTok{(x,}\DecValTok{4}\NormalTok{))}
\KeywordTok{summary}\NormalTok{(modelc)}
\end{Highlighting}
\end{Shaded}

\begin{verbatim}
## 
## Time series regression with "ts" data:
## Start = 1964, End = 2010
## 
## Call:
## dynlm(formula = y ~ L(x, 1) + L(x, 2) + L(x, 3) + L(x, 4))
## 
## Residuals:
##      Min       1Q   Median       3Q      Max 
## -12.8470  -3.6448  -0.1239   3.9760  11.5391 
## 
## Coefficients:
##             Estimate Std. Error t value Pr(>|t|)  
## (Intercept)  0.26221    0.81984   0.320   0.7507  
## L(x, 1)      0.12840    0.04971   2.583   0.0134 *
## L(x, 2)     -0.05268    0.05695  -0.925   0.3602  
## L(x, 3)     -0.02281    0.05563  -0.410   0.6838  
## L(x, 4)      0.02086    0.05375   0.388   0.6998  
## ---
## Signif. codes:  0 '***' 0.001 '**' 0.01 '*' 0.05 '.' 0.1 ' ' 1
## 
## Residual standard error: 5.287 on 42 degrees of freedom
## Multiple R-squared:  0.1596, Adjusted R-squared:  0.07957 
## F-statistic: 1.994 on 4 and 42 DF,  p-value: 0.1129
\end{verbatim}

\indent For model c, multiple R-squared is 0.1596, adjusted R-squared is
0.07957.

\begin{Shaded}
\begin{Highlighting}[]
\CommentTok{# (d) }
\NormalTok{modeld=}\KeywordTok{dynlm}\NormalTok{(y}\OperatorTok{~}\KeywordTok{L}\NormalTok{(x,}\DecValTok{1}\NormalTok{)}\OperatorTok{+}\KeywordTok{L}\NormalTok{(x,}\DecValTok{2}\NormalTok{)}\OperatorTok{+}\KeywordTok{L}\NormalTok{(x,}\DecValTok{3}\NormalTok{)}\OperatorTok{+}\KeywordTok{L}\NormalTok{(x,}\DecValTok{4}\NormalTok{)}\OperatorTok{+}\KeywordTok{L}\NormalTok{(y,}\DecValTok{1}\NormalTok{))}
\KeywordTok{summary}\NormalTok{(modeld)}
\end{Highlighting}
\end{Shaded}

\begin{verbatim}
## 
## Time series regression with "ts" data:
## Start = 1964, End = 2010
## 
## Call:
## dynlm(formula = y ~ L(x, 1) + L(x, 2) + L(x, 3) + L(x, 4) + L(y, 
##     1))
## 
## Residuals:
##     Min      1Q  Median      3Q     Max 
## -9.9066 -2.6588  0.2105  3.1978  8.1328 
## 
## Coefficients:
##             Estimate Std. Error t value Pr(>|t|)    
## (Intercept)  0.67751    0.72750   0.931 0.357157    
## L(x, 1)     -0.01948    0.05918  -0.329 0.743647    
## L(x, 2)     -0.07897    0.05043  -1.566 0.125031    
## L(x, 3)     -0.02047    0.04877  -0.420 0.676876    
## L(x, 4)      0.00998    0.04722   0.211 0.833648    
## L(y, 1)      0.67568    0.18293   3.694 0.000646 ***
## ---
## Signif. codes:  0 '***' 0.001 '**' 0.01 '*' 0.05 '.' 0.1 ' ' 1
## 
## Residual standard error: 4.635 on 41 degrees of freedom
## Multiple R-squared:  0.3694, Adjusted R-squared:  0.2925 
## F-statistic: 4.804 on 5 and 41 DF,  p-value: 0.001509
\end{verbatim}

\indent For model d, multiple R-squared is 0.3694, adjusted R-squared is
0.2925.

\indent <br> By comparing adjusted R-squared, I prefer model a.
\indent Compare AIC and BIC:

\begin{Shaded}
\begin{Highlighting}[]
\KeywordTok{AIC}\NormalTok{(modela,modelb,modelc,modeld)}
\end{Highlighting}
\end{Shaded}

\begin{verbatim}
## Warning in AIC.default(modela, modelb, modelc, modeld): models are not all
## fitted to the same number of observations
\end{verbatim}

\begin{verbatim}
##        df      AIC
## modela  3 285.5185
## modelb  3 309.3521
## modelc  6 296.6272
## modeld  7 285.1260
\end{verbatim}

\begin{Shaded}
\begin{Highlighting}[]
\KeywordTok{BIC}\NormalTok{(modela,modelb,modelc,modeld)}
\end{Highlighting}
\end{Shaded}

\begin{verbatim}
## Warning in BIC.default(modela, modelb, modelc, modeld): models are not all
## fitted to the same number of observations
\end{verbatim}

\begin{verbatim}
##        df      BIC
## modela  3 291.3139
## modelb  3 315.0881
## modelc  6 307.7281
## modeld  7 298.0771
\end{verbatim}

\indent If I select model with respect to lowest AIC/BIC, I will prefer
model a.


\end{document}
